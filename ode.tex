\documentclass[a4paper]{article}

\title{Efficient numerical methods to solve ballistic equations}
\author{Javier Burguete}
\date{\today}

\newcommand{\ABS}[1]{\left|#1\right|}
\newcommand{\C}[1]{\left[#1\right]}
\newcommand{\COMB}[2]{\left(\begin{array}{c}#1\\#2\end{array}\right)}
\newcommand{\D}[3]{\DD{^{#3}#1}{#2^{#3}}}
\newcommand{\DD}[2]{\frac{d#1}{d#2}}
\newcommand{\EQ}[2]{\begin{equation}#1\label{#2}\end{equation}}
\newcommand{\PA}[1]{\left(#1\right)}
\newcommand{\PARTIAL}[2]{\frac{\partial #1}{\partial #2}}

\begin{document}

\maketitle

\tableofcontents

\section{Introduction}

We are interested in solving ordinary differential equations (ODE) as:
\EQ{\DD{u}{t}=f\PA{t,\,u(t)}.}{EqODE}

The following notation has been used in this work:
\[u_0=u\PA{t_0},\quad f_0=f\PA{t_0,\,u\PA{t_0}},\]
\[
	f_{n\,t}=\PARTIAL{^nf}{t^n},\quad
	f_{n\,t,m\,u}=\PARTIAL{^{n+m}f}{t^n\,\partial u^m},
\]
\[F_{0,0}=f_0,\quad F_{1,0}=\PA{f_t+f\,f_u}_0,\quad F_{1,1}=\PA{f_u}_0,\]
\[
	F_{2,0}=\PA{f_{2t}+2\,f\,f_{t,u}+f^2\,f_{2u}}_0,\quad
	F_{2,1}=\PA{f_{t,u}+f\,f_{2u}}_0,\quad
	F_{2,2}=\PA{f_{2u}}_0,
\]
\[\cdots\]
\[
	F_{i,j}=\C{\sum_{k=0}^{i-j}\COMB{i-j}{k}\,f^k\,f_{(i-j-k)\,t,(j+k)\,u}}_0,
	\quad j\leq i.
\]

The following recurrence rule is hold:
\EQ{\DD{F_{i,j}}{t}=F_{i+1,j}+(i-j)\,F_{i,j+1}\,F_{1,0}.}{EqdF}

Then,
\[\left.\DD{u}{t}\right|_0=F_{0,0},\]
\[\left.\D{u}{t}{2}\right|_0=F_{1,0},\]
\[\left.\D{u}{t}{3}\right|_0=F_{2,0}+F_{1,1}\,F_{1,0},\]
\[
	\left.\D{u}{t}{4}\right|_0=F_{3,0}+3\,F_{2,1}\,F_{1,0}+F_{2,0}\,F_{1,1}
	+F_{1,1}^2\,F_{1,0},
\]
\[
	\left.\D{u}{t}{5}\right|_0=F_{4,0}+6\,F_{3,1}\,F_{1,0}+F_{3,0}\,F_{1,1}
	+3\,F_{2,2}\,F_{1,0}^2+4\,F_{2,1}\,F_{2,0}+7\,F_{2,1}\,F_{1,1}\,F_{1,0}
\]
\[
	+F_{2,0}\,F_{1,1}^2+F_{1,1}^3\,F_{1,0},
\]

When $f=f(t)$ is a function depending only in time, the above relatons are
considerably simplified:
\[
	\left.\D{u}{t}{n}\right|_0=F_{n,0};\quad
	F_{i,j}=0,\;\forall j>0.
\]

Numerical methods to solve ODE (\ref{EqODE}) are based in approximations of the 
Taylor's serie. For instance, in 5-th order:
\[
	u\PA{t_0+h}=u_0+h\,\left.\DD{u}{t}\right|_0
	+\frac{h^2}{2}\,\left.\D{u}{t}{2}\right|_0
	+\frac{h^3}{6}\,\left.\D{u}{t}{3}\right|_0
	+\frac{h^4}{24}\,\left.\D{u}{t}{4}\right|_0
	+\frac{h^5}{120}\,\left.\D{u}{t}{5}\right|_0
\]
\[+O\PA{h^6}\]
\[
	=u_0+h\,F_{0,0}+\frac{h^2}{2}\,F_{1,0}
	+\frac{h^3}{6}\,\PA{F_{2,0}+F_{1,1}\,F_{1,0}}
\]
\[
	+\frac{h^4}{24}\,\PA{F_{3,0}+3\,F_{2,1}\,F_{1,0}+F_{2,0}\,F_{1,1}
	+F_{1,1}^2\,F_{1,0}}
\]
\[
	+\frac{h^5}{120}\,\left(F_{4,0}+6\,F_{3,1}\,F_{1,0}+F_{3,0}\,F_{1,1}
	+3\,F_{2,2}\,F_{1,0}^2+4\,F_{2,1}\,F_{2,0}+7\,F_{2,1}\,F_{1,1}\,F_{1,0}
	\right.
\]
\EQ{\left.+F_{2,0}\,F_{1,1}^2+F_{1,1}^3\,F_{1,0}\right)+O\PA{h^6}.}{EqTaylor}

\subsection{Decay ODE}

A common particular case of ODE are the ''decay equations'' defined as:
\EQ{\DD{u}{t}=-|g(u)|\,u,\quad g_u\geq0.}{EqDODE}
with $g$ an increasing function on $u$. In these equations the variable module
decreases always in time:
\[\frac{du}{u}=-|g|\,dt\Rightarrow\]
\[u(t)=u_0\,\exp\PA{-\int_0^t|g|\,dt}\Rightarrow\]
\EQ{|u(t)|=\ABS{u_0}\,\exp\PA{-\int_0^t|g|\,dt.}}{EqDecay}
As $|g|$ is positive the integral is an increasing function. Then, $|u|$ is a
decreasing function. Another interesting property of these equations is that $u$
cannot change the sign:
\EQ{\frac{u(t)}{u_0}=\exp\PA{-\int_0^t|g|\,dt}\geq 0.}{EqDODEP}
Solving DODE (\ref{EqDODE}) numerically in first order using the Euler's rule:
\[u\PA{t_0+h}=u\PA{t_0}-h\,\ABS{g\PA{u\PA{t_0}}}\,u\PA{t_0}\Rightarrow\]
\EQ{\frac{u\PA{t_0+h}}{u\PA{t_0}}=1-h\,\ABS{g\PA{u\PA{t_0}}}.}{EqDODEI}
Then, to prevent numerical variable sign changes the time step size has to be
limited by:
\EQ
{
	\frac{u\PA{t_0+h}}{u\PA{t_0}}\geq0\Rightarrow\;
	1-h\,\ABS{g\PA{u\PA{t_0}}}\geq0\Rightarrow\;
	h\leq\frac{1}{|g|}.
}{EqDODELimit}
In these equations, we use the time step size as:
\EQ{h=\frac{k_t}{|g|},}{EqDODE dt}
with $k_t\leq1$ the dimensionless stability factor. 

\subsection{Ballistic ODE}

Ballistic equations are a particular form of ODE follow a form:
\EQ{\ddot{\vec{r}}(t)=\ddot{\vec{r}}\PA{\dot{\vec{r}}(t)},}{EqBODE}
where $\vec{r}$, $\dot{\vec{r}}$, $\ddot{\vec{r}}$ are the vectors of position,
velocity and acceleration respectively.

\section{Runge-Kutta methods}

We use a variation of the Einstein's notation, where repeated indeces are
summed, in order to simplify the notation:
\EQ{f_{i,j}\,g_j\equiv\sum_{j=0}^{i-1}f_{i,j}\,g_j.}{EqNotation}
Runge-Kutta methods are methods in the form:
\EQ
{
	u_i=u_0+h\,b_{i,j}\,f_j,\quad
	f_i=f\PA{t+d_i\,h,u_i},\quad
	d_0=0,\quad
	d_i=\sum_{j=0}^{i-1}b_{i,j}.
}{EqRK}
Every $i$-th step introduces $i$ new freedom degrees.

Developing the steps in Taylor's series to 5-th order:
\[u_1=u_0+h\,d_1\,F_{0,0},\]
\[
	f_1=F_{0,0}+h\,d_1\,F_{1,0}+\frac{h^2\,d_1^2}{2}\,F_{2,0}
	+\frac{h^3\,d_1^3}{6}\,F_{3,0}+\frac{h^4\,d_1^4}{24}\,F_{4,0}+O\PA{h^5},
\]
\[
	u_2=u_0+h\,d_2\,F_{0,0}+h^2\,b_{2,j}\,d_j\,F_{1,0}
	+\frac{h^3\,b_{2,j}\,d_j^2}{2}\,F_{2,0}
	+\frac{h^4\,b_{2,j}\,d_j^3}{6}\,F_{3,0}
\]
\[	+\frac{h^5\,b_{2,j}\,d_j^4}{24}\,F_{4,0}+O\PA{h^6},\]
\[
	f_2=F_{0,0}+h\,d_2\,F_{1,0}+\frac{h^2\,d_2^2}{2}\,F_{2,0}
	+h^2\,b_{2,j}\,d_j\,F_{1,1}\,F_{1,0}+\frac{h^3\,d_2^3}{6}\,F_{3,0}
\]
\[
	+\frac{h^3\,b_{2,j}\,d_j^2}{2}\,F_{2,0}\,F_{1,1}
	+\frac{h^4\,d_2^4}{24}\,F_{4,0}
	+\frac{h^4\,d_2^2\,b_{2,j}\,d_j}{2}\,F_{3,1}\,F_{1,0}
\]
\[
	+\frac{h^4\,b_{2,j}\,d_j^3}{6}\,F_{3,0}\,F_{1,1}
	+\frac{h^4\,\PA{b_{2,j}\,d_j}^2}{2}\,F_{2,2}\,F_{1,0}^2
	+\frac{h^4\,d_2\,b_{2,j}\,d_j^2}{2}\,F_{2,1}\,F_{2,0}
	+O\PA{h^5},
\]
\[
	u_3=u_0+h\,d_3\,F_{0,0}+h^2\,b_{3,j}\,d_j\,F_{1,0}
	+\frac{h^3\,b_{3,j}\,d_j^2}{2}\,F_{2,0}
	+h^3\,b_{3,j}\,b_{j,k}\,d_k\,F_{1,1}\,F_{1,0}
\]
\[
	+\frac{h^4\,b_{3,j}\,d_j^3}{6}\,F_{3,0}
	+\frac{h^4\,b_{3,j}\,b_{j,k}\,d_k^2}{2}\,F_{2,0}\,F_{1,1}
	+\frac{h^5\,b_{3,j}\,d_j^4}{24}\,F_{4,0}
\]
\[
	+\frac{h^5\,b_{3,j}\,d_j^2\,b_{j,k}\,d_k}{2}\,F_{3,1}\,F_{1,0}
	+\frac{h^5\,b_{3,j}\,b_{j,k}\,d_k^3}{6}\,F_{3,0}\,F_{1,1}
	+\frac{h^5\,b_{3,j}\,\PA{b_{j,k}\,d_k}^2}{2}\,F_{2,2}\,F_{1,0}^2
\]
\[
	+\frac{h^5\,b_{3,j}\,d_j\,b_{j,k}\,d_k^2}{2}\,F_{2,1}\,F_{2,0}
	+O\PA{h^6}
\]
\[
	f_3=F_{0,0}+h\,d_3\,F_{1,0}+\frac{h^2\,d_3^2}{2}\,F_{2,0}
	+h^2\,b_{3,j}\,d_j\,F_{1,1}\,F_{1,0}+\frac{h^3\,d_3^3}{6}\,F_{3,0}
\]
\[
	+h^3\,d_3\,b_{3,j}\,d_j\,F_{2,1}\,F_{1,0}
	+\frac{h^3\,b_{3,j}\,d_j^2}{2}\,F_{2,0}\,F_{1,1}
	+h^3\,b_{3,j}\,b_{j,k}\,d_k\,F_{1,1}^2\,F_{1,0}
	+\frac{h^4\,d_3^4}{24}\,F_{4,0}
\]
\[
	+\frac{h^4\,d_3^2\,b_{3,j}\,d_j}{2}\,F_{3,1}\,F_{1,0}
	+\frac{h^4\,b_{3,j}\,d_j^3}{6}\,F_{3,0}\,F_{1,1}
	+\frac{h^4\,\PA{b_{3,j}\,d_j}^2}{2}\,F_{2,2}\,F_{1,0}^2
\]
\[
	+\frac{h^4\,d_3\,b_{3,j}\,d_j^2}{2}\,F_{2,1}\,F_{2,0}
	+h^4\,d_3\,b_{3,j}\,b_{j,k}\,d_k\,F_{2,1}\,F_{1,1}\,F_{1,0}
	+\frac{h^4\,b_{3,j}\,b_{j,k}\,d_k^2}{2}\,F_{2,0}\,F_{1,1}^2
\]
\[+O\PA{h^5},\]
\[
	u_4=u_0+h\,d_4\,F_{0,0}+h^2\,b_{4,j}\,d_j\,F_{1,0}
	+\frac{h^3\,b_{4,j}\,d_j^2}{2}\,F_{2,0}
	+h^3\,b_{4,j}\,b_{j,k}\,d_k\,F_{1,1}\,F_{1,0}
\]
\[
	+\frac{h^4\,b_{4,j}\,d_j^3}{6}\,F_{3,0}
	+h^4\,b_{4,j}\,d_j\,b_{j,k}\,d_k\,F_{2,1}\,F_{1,0}
	+\frac{h^4\,b_{4,j}\,b_{j,k}\,d_k^2}{2}\,F_{2,0}\,F_{1,1}
\]
\[
	+h^4\,b_{4,j}\,b_{j,k}\,b_{k,l}\,d_l\,F_{1,1}^2\,F_{1,0}
	+\frac{h^5\,b_{4,j}\,d_j^4}{24}\,F_{4,0}
	+\frac{h^5\,b_{4,j}\,d_j^2\,b_{j,k}\,d_k}{2}\,F_{3,1}\,F_{1,0}
\]
\[
	+\frac{h^5\,b_{4,j}\,b_{j,k}\,d_k^3}{6}\,F_{3,0}\,F_{1,1}
	+\frac{h^5\,b_{4,j}\,\PA{b_{j,k}\,d_k}^2}{2}\,F_{2,2}\,F_{1,0}^2
	+\frac{h^5\,b_{4,j}\,d_j\,b_{j,k}\,d_k^2}{2}\,F_{2,1}\,F_{2,0}
\]
\[
	+h^5\,b_{4,j}\,d_j\,b_{j,k}\,b_{k,l}\,d_l\,F_{2,1}\,F_{1,1}\,F_{1,0}
	+\frac{h^5\,b_{4,j}\,b_{j,k}\,b_{k,l}\,d_k^2}{2}\,F_{2,0}\,F_{1,1}^2
	+O\PA{h^6}
\]

Comparing with the Taylor's series (\ref{EqTaylor}) a $i$-steps Runge-Kutta
method has to hold the following conditions to get different approximation
orders:
\begin{itemize}
\item First order:
	\EQ{d_i=1.}{EqRKI}
\item Second order:
	\EQ{b_{i,j}\,d_j=\frac12.}{EqRKII}
\item Third order:
	\EQ
	{
		b_{i,j}\,d_j^2=\frac13,\quad
		b_{i,j}\,b_{j,k}\,d_k=\frac16.
	}{EqRKIII}
\item Fourth order:
	\EQ
	{
		b_{i,j}\,d_j^3=\frac14,\quad
		b_{i,j}\,d_j\,b_{j,k}\,d_k=\frac18,\quad
		b_{i,j}\,b_{j,k}\,d_k^2=\frac1{12},\quad
		b_{i,j}\,b_{j,k}\,b_{k,l}\,d_l=\frac1{24}.
	}{EqRKIV}
\item Fifth order:
	\[
		b_{i,j}\,d_j^4=\frac15,\quad
		b_{i,j}\,d_j^2\,b_{j,k}\,d_k=\frac1{10},\quad
		b_{i,j}\,b_{j,k}\,d_k^3=\frac1{20},
	\]
	\[
		b_{i,j}\,\PA{b_{j,k}\,d_k}^2=\frac1{20},\quad
		b_{i,j}\,d_j\,b_{j,k}\,d_k^2=\frac1{15},\quad
		b_{i,j}\,d_j\,b_{j,k}\,b_{k,l}\,d_l=\frac7{120},
	\]
	\EQ
	{
		b_{i,j}\,b_{j,k}\,b_{k,l}\,d_l^2=\frac1{60},\quad
		b_{i,j}\,b_{j,k}\,b_{k,l}\,b_{l,m}\,d_m=\frac1{120}.
	}{EqRKV}
\end{itemize}

On equations depending only in time, first of the above equations are
sufficient:
\begin{itemize}
\item $n$-th order:
	\EQ{b_{i,j}\,d_j^{n-1}=\frac1n.}{EqRKt}
\end{itemize}

\subsection{Strong stability}

General Runge-Kutta methods (\ref{EqRK}) can also be expressed as:
\EQ
{
	u_i=\sum_{j=0}^{i-1}a_{i,j}\,\PA{u_j+h\,c_{i,j}\,f_j},\quad
	\sum_{j=0}^{i-1}a_{i,j}=1.
}{EqRKac}
The following relation is hold:
\EQ{b_{i,j}=(a\,c)_{i,j}+\sum_{k=j+1}^{i-1}a_{i,k}\,b_{k,j}.}{EqRKacb}
This form introduces new $i-1$ freedom degrees for each $i$-th step.

In the context of propagation equations it has been demonstrated that a
Runge-Kutta method preserves monotonicity, and therefore absence of numerical
spurious oscillations, if:
\EQ{a_{i,j},\,c_{i,j}\geq0,\quad k_t\leq\frac1{\max\PA{c_{i,j}}}}{EqRKTVD}

\subsection{Simple stability}

On decay ODE a more simple stability condition can be obtained. In these
equations we define a Runge-Kutta method as ''simple stable'' when it preserves
the variable sign for each time step. Assuming positive $u$:
\[
	b_{i,j}\geq0\Rightarrow\;
	u_i=u_0-h\,b_{i,j}\,\ABS{g_j}\,u_j\leq u_0\Rightarrow\;
	f_i\leq f_0\Rightarrow
\]
\EQ
{
	u_i\geq u_0-h\,\sum_jb_{i,j}\,\ABS{g_0}\,u_0
	=u_0\,\PA{1-h\,d_i\,\ABS{g_0}}
}{EqRKSimpleCondition}
wherefrom preserving sign in $u$ is ensured by:
\EQ{b_{i,j}\geq0,\quad k_t\leq\frac{1}{\max\PA{d_i}}}{EqRKSimple}
These conditions are sufficient but not necessary.

A strong stable method is always simple stable, (\ref{EqRKTVD}) conditions
ensures:
\[\frac{u_j+h\,c_{i,j}\,f_j}{u_j}\geq0\Rightarrow\]
\EQ{\frac{u_i}{u_0}=\sum_ja_{i,j}\,\frac{u_j+h\,c_{i,j}\,f_j}{u_j}\geq0.}
{EqRKSimpleStrong}

\subsection{Ballistic ODE}

For ballistic ODE, Runge-Kutta methods are adapted by doing:
\EQ
{
	\vec{r}_i=\vec{r}_0+h\,b_{i,j}\,\dot{\vec{r}}_j,\quad
	\dot{\vec{r}}_i=\dot{\vec{r}}_0+h\,b_{i,j}\,\ddot{\vec{r}}_j,\quad
	\ddot{\vec{r}}_i=\ddot{\vec{r}}\PA{\dot{\vec{r}}_i}.
}{EqRKBODE}

\subsection{One step first order (Euler's method)}

The one step first order method, proposed by Euler, is defined by the
coefficient matrices of table~\ref{TableRKiI}.
\begin{table}[ht]
	\centering
	\begin{tabular}{c|c}
		$d_{i}$ & $b_{i,j}$ \\ \hline
		1 & 1
	\end{tabular},~
	\begin{tabular}{c}
		$a_{i,j}$ \\ \hline
		1
	\end{tabular},~
	\begin{tabular}{c}
		$c_{i,j}$ \\ \hline
		1
	\end{tabular}
	\caption{Coefficients of the one step first order Runge-Kutta method (RK1-1).
		\label{TableRKiI}}
\end{table}
This a first order method with strong and simple stability for $k_t\leq1$.

\subsection{Two steps second order}

We consider two interesting two steps second order methods defined by the
coefficients matrices of tables~\ref{TableRKiiIIa} and~\ref{TableRKiiIIb}.
First method is second order, third order in equations depending only in time,
simple stable for $k_t\leq1$ and strong stable for $k_t\leq1/2$. Second method
is second order and simple and strong stable for $k_t\leq1$.
\begin{table}[ht]
	\centering
	\begin{tabular}{c|cc}
		$d_{i}$ & $b_{i,j}$ \\ \hline
		2/3 & 2/3 \\
		1 & 1/4 & 3/4
	\end{tabular},~
	\begin{tabular}{cc}
		$a_{i,j}$ \\ \hline
		1 \\
		5/8 & 3/8
	\end{tabular},~
	\begin{tabular}{cc}
		$c_{i,j}$ \\ \hline
		2/3 \\
		0 & 2
	\end{tabular}
	\caption{Coefficients of the two steps second order Runge-Kutta method optime
		in accuracy (RK2-2(3)).\label{TableRKiiIIa}}
\end{table}
\begin{table}[ht]
	\centering
	\begin{tabular}{c|cc}
		$d_{i}$ & $b_{i,j}$ \\ \hline
		1 & 1 \\
		1 & 1/2 & 1/2
	\end{tabular},~
	\begin{tabular}{cc}
		$a_{i,j}$ \\ \hline
		1 \\
		1/2 & 1/2
	\end{tabular},~
	\begin{tabular}{cc}
		$c_{i,j}$ \\ \hline
		1 \\
		0 & 1
	\end{tabular}
	\caption{Coefficients of the two steps second order Runge-Kutta method optime
		in strong stability (RK2-2).\label{TableRKiiIIb}}
\end{table}

\subsection{Three steps third order}

The most interesting three steps Runge-Kutta is the defined by the coefficients
of the table~\ref{TableRKiiiIII}.
This method is third order, fourth order in equations depending only in time and
simple and strong stable for $k_t\leq1$.
\begin{table}[ht]
	\centering
	\begin{tabular}{c|ccc}
		$d_{i}$ & $b_{i,j}$ \\ \hline
		1 & 1 \\
		1/2 & 1/4 & 1/4 \\
		1 & 1/6 & 1/6 & 2/3
	\end{tabular},~
	\begin{tabular}{ccc}
		$a_{i,j}$ \\ \hline
		1 \\
		3/4 & 1/4 \\
		1/3 & 0 & 2/3
	\end{tabular},~
	\begin{tabular}{ccc}
		$c_{i,j}$ \\ \hline
		1 \\
		0 & 1 \\
		0 & 0 & 1
	\end{tabular}
	\caption{Coefficients of the three steps third order Runge-Kutta method
		(RK3-3(4)).\label{TableRKiiiIII}}
\end{table}

\subsection{Four steps fourth order}

The only four steps fourth order simple stable for $k_t\leq1$ is the defined by
the coefficients of the table~\ref{TableRKivIV}.
They are not fourth order strong stable Runge-Kutta methods of four steps.
\begin{table}[ht]
	\centering
	\begin{tabular}{c|cccc}
		$d_{i}$ & $b_{i,j}$ \\ \hline
		1/2 & 1/2 \\
		1/2 & 0 & 1/2 \\
		1 & 0 & 0 & 1 \\
		1 & 1/6 & 1/3 & 1/3 & 1/6
	\end{tabular}
	\caption{Coefficients of the four steps fourth order Runge-Kutta method
		(RK4-4).\label{TableRKivIV}}
\end{table}

\subsection{Five steps fourth order}

It is not possible to build a fifth order Runge-Kutta method with five steps.
The method defined in table~\ref{TableRKvIV} is simple stable for $k_t\leq1$,
fourth order and fifth order in equations depending only on time.
\begin{table}[ht]
	\renewcommand*{\arraystretch}{1.3}
	\centering
	\begin{tabular}{c|ccccc}
		$d_{i}$ & $b_{i,j}$ \\ \hline
		$\frac{11}{40}$ & $\frac{11}{40}$ \\
		1 & $\frac{223}{803}$ & $\frac{580}{803}$ \\
		$\frac12$ & $\frac{1.41}{146}$ & $\frac25$ & $\frac{13.19}{146}$ \\
		$\frac{17}{20}$ & $\frac{661.15057}{19272}$ & 
			$\frac{11289.7731}{46574}$ & $\frac{156.50557}{50808}$
			& $\frac{2737}{4800}$ \\
		1 & $\frac{107}{1122}$ & $\frac{64000}{198099}$ & $\frac7{522}$ & 
			$\frac{46}{189}$ & $\frac{8000}{24633}$
	\end{tabular}
	\caption{Coefficients of the five steps fourth order, fifth order in
		equations depending only in time, Runge-Kutta method (RK5-4(5)).
		\label{TableRKvIV}}
\end{table}

\section{Runge-Kutta pairs}

\EQ{\hat{u}_i=u_0+h\,\hat{b}_{i,j}\,f_j.}{EqRKpair}

\EQ{E_i=\hat{u}_i-u_i=h\,\PA{\hat{b}-b}_{i,j}\,f_j.}{EqRKpairE}

\subsection{Two steps first-second order pairs}

The two second order methods defined by the coefficients matrices of
tables~\ref{TableRKiiIIa} and~\ref{TableRKiiIIb} can be combined con the
Euler's method defined in table~\ref{TableRKiI} to build the two steps second
order Runge-Kutta pairs defined in tables~\ref{TableRKIiiIIa}
and~\ref{TableRKIiiIIb}.
\begin{table}[ht]
	\centering
	\begin{tabular}{c|cc}
		$d_{i}$ & $b_{i,j}$ \\ \hline
		2/3 & 2/3 \\
		1 & 1/4 & 3/4 \\ \hline
		$\hat{b}_{2,j}$ & 1 & 0 \\ \hline
		$e_{2,j}$ & 3/4 & -3/4
	\end{tabular}
	\caption{Coefficients of the two steps first-second order, first-third order
		in equations depending only in time, Runge-Kutta pair (RK2-1-2(3)).
		\label{TableRKIiiIIa}}
\end{table}
\begin{table}[ht]
	\centering
	\begin{tabular}{c|cc}
		$d_{i}$ & $b_{i,j}$ \\ \hline
		1 & 1 \\
		1 & 1/2 & 1/2 \\ \hline
		$\hat{b}_{2,j}$ & 1 & 0 \\ \hline
		$e_{2,j}$ & 1/2 & -1/2
	\end{tabular}
	\caption{Coefficients of the two steps first-second order Runge-Kutta pair
		optime in strong stability (RK2-1-2).\label{TableRKIiiIIb}}
\end{table}

\subsection{Three steps second-third order pair}

The three steps third order Runge-Kutta defined in table~\ref{TableRKiiiIII}
can be combined with the second order defined in table~\ref{TableRKiiIIb} to
produce the three steps second-third order pair defined in
table~\ref{TableRKIIiiiIII}.
\begin{table}[ht]
	\centering
	\begin{tabular}{c|ccc}
		$d_{i}$ & $b_{i,j}$ \\ \hline
		1 & 1 \\
		1/2 & 1/4 & 1/4 \\
		1 & 1/6 & 1/6 & 2/3 \\ \hline
		$\hat{b}_{3,j}$ & 1/2 & 1/2 & 0 \\ \hline
		$e_{3,j}$ & 1/3 & 1/3 & -2/3
	\end{tabular}
	\caption{Coefficients of the three steps second-third order, second-fourth
		order in equations depending only in time, Runge-Kutta pair (RK3-2-3(4)).
		\label{TableRKIIiiiIII}}
\end{table}

\subsection{Four steps second-third order pair}

It is not possible to build a four steps third-fourth order pair but they are
infinity two-three order pairs of four steps. The defined in
table~\ref{TableRKIIivIII} is moreover simple stable for $k_t\leq1$ and 
third-fourth order in equations depending only on time.
\begin{table}[ht]
	\renewcommand*{\arraystretch}{1.3}
	\centering
	\begin{tabular}{c|cccc}
		$d_{i}$ & $b_{i,j}$ \\ \hline
		$\frac13$ & $\frac13$ \\
		1 & 0 & 1 \\
		$\frac23$ & $\frac16$ & $\frac14$ & $\frac14$ \\
		1 & $\frac18$ & $\frac38$ & $\frac18$ & $\frac38$ \\ \hline
		$\hat{b}_{4,j}$ & 0 & $\frac34$ & $\frac14$ & 0 \\ \hline
		$e_{4,j}$ & $-\frac18$ & $\frac38$ & $\frac18$ & $-\frac38$
	\end{tabular}
	\caption{Coefficients of the four steps second-third order, third-fourth order
		in equations depending only in time, Runge-Kutta pair (RK4-2(3)-3(4)).
		\label{TableRKIIivIII}}
\end{table}

\subsection{Five steps third-fourth order pair}

They are infinity third-fourth order pairs of five steps. The defined in
table~\ref{TableRKIIIvIV} is moreover simple stable for $k_t\leq1$ and 
fourth-fifth order in equations depending only on time.
\begin{table}[ht]
	\renewcommand*{\arraystretch}{1.3}
	\centering
	\begin{tabular}{c|ccccc}
		$d_{i}$ & $b_{i,j}$ \\ \hline
		$\frac{11}{40}$ & $\frac{11}{40}$ \\
		1 & $\frac{223}{803}$ & $\frac{580}{803}$ \\
		$\frac12$ & $\frac{141}{14600}$ & $\frac25$ & $\frac{1319}{14600}$ \\
		$\frac{17}{20}$ & $\frac{66115057}{1927200000}$ & 
			$\frac{112897731}{465740000}$ & $\frac{15650557}{5080800000}$
			& $\frac{2737}{4800}$ \\
		1 & $\frac{107}{1122}$ & $\frac{64000}{198099}$ & $\frac7{522}$ & 
			$\frac{46}{189}$ & $\frac{8000}{24633}$ \\ \hline
		$\hat{b}_{5,j}$ & $\frac16$ & 0 & $\frac16$ & $\frac23$ & 0 \\ \hline
		$e_{5,j}$ & $\frac{40}{561}$ & $-\frac{64000}{198099}$ & $\frac{40}{261}$
			& $\frac{80}{189}$ & $-\frac{8000}{24633}$
	\end{tabular}
	\caption{Coefficients of the five steps third-fourth order, fourth-fifth order
		in equations depending only in time, Runge-Kutta pair (RK-3(4)-4(5)).
		\label{TableRKIIIvIV}}
\end{table}

\section{Multi-steps methods}

\section{Multi-steps pairs}

\section{Tests}

\section{Conclusions}

\end{document}
