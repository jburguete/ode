\documentclass[a4paper]{article}

\title{Numerical methods to solve ordinary differential equations}
\author{Javier Burguete}
\date{\today}

\newcommand{\C}[1]{\left[#1\right]}
\newcommand{\COMB}[2]{\left(\begin{array}{c}#1\\#2\end{array}\right)}
\newcommand{\D}[3]{\DD{^{#3}#1}{#2^{#3}}}
\newcommand{\DD}[2]{\frac{d#1}{d#2}}
\newcommand{\EQ}[2]{\begin{equation}#1\label{#2}\end{equation}}
\newcommand{\PA}[1]{\left(#1\right)}
\newcommand{\PARTIAL}[2]{\frac{\partial #1}{\partial #2}}

\begin{document}

\maketitle

\tableofcontents

\section{Introduction}

We are interested in solving ordinary differential equations (ODE) as:
\EQ{\DD{u}{t}=f\PA{t,\,u(t)}.}{EqODE}

The following notation has been used in this work:
\[u_0=u\PA{t_0},\quad f_0=f\PA{t_0,\,u\PA{t_0}},\]
\[
	f_{n\,t}=\PARTIAL{^nf}{t^n},\quad
	f_{n\,t,m\,u}=\PARTIAL{^{n+m}f}{t^n\,\partial u^m},
\]
\[F_{0,0}=f_0,\quad F_{1,0}=\PA{f_t+f\,f_u}_0,\quad F_{1,1}=\PA{f_u}_0,\]
\[
	F_{2,0}=\PA{f_{2t}+2\,f\,f_{t,u}+f^2\,f_{2u}}_0,\quad
	F_{2,1}=\PA{f_{t,u}+f\,f_{2u}}_0,\quad
	F_{2,2}=\PA{f_{2u}}_0,
\]
\[\cdots\]
\[
	F_{i,j}=\C{\sum_{k=0}^{i-j}\COMB{i-j}{k}\,f^k\,f_{(i-j-k)\,t,(j+k)\,u}}_0,
	\quad j\leq i.
\]

The following recurrence rule is hold:
\EQ{\DD{F_{i,j}}{t}=F_{i+1,j}+(i-j)\,F_{i,j+1}\,F_{1,0}.}{EqdF}

Then,
\[\left.\DD{u}{t}\right|_0=F_{0,0},\]
\[\left.\D{u}{t}{2}\right|_0=F_{1,0},\]
\[\left.\D{u}{t}{3}\right|_0=F_{2,0}+F_{1,1}\,F_{1,0},\]
\[
	\left.\D{u}{t}{4}\right|_0=F_{3,0}+3\,F_{2,1}\,F_{1,0}+F_{2,0}\,F_{1,1}
	+F_{1,1}^2\,F_{1,0},
\]
\[
	\left.\D{u}{t}{5}\right|_0=F_{4,0}+6\,F_{3,1}\,F_{1,0}+F_{3,0}\,F_{1,1}
	+3\,F_{2,2}\,F_{1,0}^2+4\,F_{2,1}\,F_{2,0}+7\,F_{2,1}\,F_{1,1}\,F_{1,0}
\]
\[
	+F_{2,0}\,F_{1,1}^2+F_{1,1}^3\,F_{1,0},
\]

Numerical methods to solve ODE (\ref{EqODE}) are based in approximate the 
Taylor's serie. For instance, in 5-th order:
\[
	u\PA{t_0+h}=u_0+h\,\left.\DD{u}{t}\right|_0
	+\frac{h^2}{2}\,\left.\D{u}{t}{2}\right|_0
	+\frac{h^3}{6}\,\left.\D{u}{t}{3}\right|_0
	+\frac{h^4}{24}\,\left.\D{u}{t}{4}\right|_0
	+\frac{h^5}{120}\,\left.\D{u}{t}{5}\right|_0
\]
\[+O\PA{h^6}\]
\[
	=u_0+h\,F_{0,0}+\frac{h^2}{2}\,F_{1,0}
	+\frac{h^3}{6}\,\PA{F_{2,0}+F_{1,1}\,F_{1,0}}
\]
\[
	+\frac{h^4}{24}\,\PA{F_{3,0}+3\,F_{2,1}\,F_{1,0}+F_{2,0}\,F_{1,1}
	+F_{1,1}^2\,F_{1,0}}
\]
\[
	+\frac{h^5}{120}\,\left(F_{4,0}+6\,F_{3,1}\,F_{1,0}+F_{3,0}\,F_{1,1}
	+3\,F_{2,2}\,F_{1,0}^2+4\,F_{2,1}\,F_{2,0}+7\,F_{2,1}\,F_{1,1}\,F_{1,0}
	\right.
\]
\EQ{\left.+F_{2,0}\,F_{1,1}^2+F_{1,1}^3\,F_{1,0}\right)+O\PA{h^6}.}{EqTaylor}

\section{Runge-Kutta methods}

Runge-Kutta methods are methods in the form:
\EQ
{
	u_i=u_0+h\,\sum_{j=0}^{i-1}b_{i,j}\,f_j,\quad
	f_i=f\PA{t+d_i\,h,u_i},\quad
	d_i=\sum_{j=0}^{i-1}b_{i,j}.
}{EqRK}

Developing the steps in Taylor's series to 5-th order:
\[u_1=u_0+h\,d_1\,F_{0,0},\]
\[
	f_1=F_{0,0}+h\,d_1\,F_{1,0}+\frac{h^2\,d_1^2}{2}\,F_{2,0}
	+\frac{h^3\,d_1^3}{6}\,F_{3,0}+\frac{h^4\,d_1^4}{24}\,F_{4,0}+O\PA{h^5},
\]
\[
	u_2=u_0+h\,d_2\,F_{0,0}+h^2\,b_{2,1}\,d_1\,F_{1,0}
	+\frac{h^3\,b_{2,1}\,d_1^2}{2}\,F_{2,0}
	+\frac{h^4\,b_{2,1}\,d_1^3}{6}\,F_{3,0}
	+\frac{h^5\,b_{2,1}\,d_1^4}{24}\,F_{4,0}
\]
\[
	+O\PA{h^6},
\]
\[
	f_2=F_{0,0}+h\,d_2\,F_{1,0}+\frac{h^2\,d_2^2}{2}\,F_{2,0}
	+h^2\,b_{2,1}\,d_{1}\,F_{1,1}\,F_{1,0}+\frac{h^3\,d_2^3}{6}\,F_{3,0}
\]
\[
	+\frac{h^3\,b_{2,1}\,d_1^2}{2}\,F_{2,0}\,F_{1,1}
	+\frac{h^4\,d_2^4}{24}\,F_{4,0}
	+\frac{h^4\,d_2^2\,b_{2,1}\,d_1}{2}\,F_{3,1}\,F_{1,0}
\]
\[
	+\frac{h^4\,b_{2,1}\,d_1^3}{6}\,F_{3,0}\,F_{1,1}
	+\frac{h^4\,\PA{b_{2,1}\,d_1}^2}{2}\,F_{2,2}\,F_{1,1}^2
	+\frac{h^4\,d_2\,b_{2,1}\,d_1^2}{2}\,F_{2,1}\,F_{2,0}
	+O\PA{h^5},
\]
\[
	u_3=u_0+h\,d_3\,F_{0,0}+h^2\,\PA{b_{3,1}\,d_1+b_{3,2}\,d_2}\,F_{1,0}
	+\frac{h^3\,\PA{b_{3,1}\,d_1^2+b_{3,2}\,d_2^2}}{2}\,F_{2,0}
\]
\[
	+h^3\,b_{3,2}\,b_{2,1}\,d_{1}\,F_{1,1}\,F_{1,0}
	+\frac{h^4\,\PA{b_{3,1}\,d_1^2+b_{3,2}\,d_2^3}}{6}\,F_{3,0}
	+\frac{h^4\,b_{3,2}\,b_{2,1}\,d_1^2}{2}\,F_{2,0}\,F_{1,1}
\]
\[
	+\frac{h^5\,\PA{b_{3,1}\,d_1^4+b_{3,2}\,d_2^4}}{24}\,F_{4,0}
	+\frac{h^5\,b_{3,2}\,d_2^2\,b_{2,1}\,d_1}{2}\,F_{3,1}\,F_{1,0}
	+\frac{h^5\,b_{3,2}\,b_{2,1}\,d_1^3}{2}\,F_{3,0}\,F_{1,1}
\]
\[
	+\frac{h^5\,b_{3,2}\,\PA{b_{2,1}\,d_1}^2}{2}\,F_{2,2}\,F_{1,0}^2
	+\frac{h^5\,b_{3,2}\,d_2\,b_{2,1}\,d_1^2}{2}\,F_{2,1}\,F_{2,0}
	+O\PA{h^6}
\]

Using the Einstein's notation, where repeated indeces are summed, and comparing
with the Taylor's series (\ref{EqTaylor}) a $i$-steps Runge-Kutta method has to
hold the following conditions to get different approximation orders:
\begin{itemize}
\item Fist order:
	\EQ{d_i=1.}{EqRKI}
\item Second order:
	\EQ{b_{i,j}\,d_j=\frac12.}{EqRKII}
\item Third order:
	\[b_{i,j}\,d_j^2=\frac13,\]
	\EQ{b_{i,j}\,b_{j,k}\,d_k=\frac16.}{EqRKIII}
\item Fourth order:
	\[b_{i,j}\,d_j^3=\frac14,\]
	\EQ{b_{i,j}\,b_{j,k}\,d_k^2=\frac1{12}.}{EqRKIV}
\end{itemize}

\subsection{Runge-Kutta pairs}

\section{Multi-steps methods}

\subsection{Multi-steps pairs}

\section{Tests}

\section{Conclusions}

\end{document}
